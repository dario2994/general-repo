\documentclass[a4paper,11pt]{article}
\usepackage{lmodern}
\renewcommand*\familydefault{\sfdefault}
\usepackage{sfmath}
\usepackage[utf8]{inputenc}
\usepackage[T1]{fontenc}
\usepackage[italian]{babel}
\usepackage{indentfirst}
\usepackage{graphicx}
\usepackage{tikz}
\newcommand*\circled[1]{\tikz[baseline=(char.base)]{
            \node[shape=circle,draw,inner sep=2pt] (char) {#1};}}
\usepackage{enumitem}
% \usepackage[group-separator={\,}]{siunitx}
\usepackage[left=2cm, right=2cm, bottom=3cm]{geometry}
\frenchspacing

\newcommand{\num}[1]{#1}

% Macro varie...
\newcommand{\file}[1]{\texttt{#1}}
\renewcommand{\arraystretch}{1.3}
\newcommand{\esempio}[2]{
\noindent\begin{minipage}{\textwidth}
\begin{tabular}{|p{11cm}|p{5cm}|}
    \hline
    \textbf{File \file{input.txt}} & \textbf{File \file{output.txt}}\\
    \hline
    \tt \small #1 &
    \tt \small #2 \\
    \hline
\end{tabular}
\end{minipage}
}

% Dati del task
\newcommand{\gara}{Gara 2 -- Stage 2}
\newcommand{\nome}{Strade in salita}
\newcommand{\nomebreve}{strade}

\begin{document}
% Intestazione
\noindent{\Large \gara}
\vspace{0.5cm}

\noindent{\Huge \textbf \nome~(\texttt{\nomebreve})}

% Descrizione del task
\section*{Descrizione del problema}
Il regno di re Pedro è costituito da $N$ meravigliose città, che
purtroppo sono irraggiungibili una dall'altra a causa dell'assenza di
strade.

Il \emph{Cipollico} sovrano, per favorire la comunicazione fra i suoi
splendenti sudditi, decide di costruire delle strade in modo che tutte
le città siano collegate. Ogni strada consente il collegamento diretto
tra due città e può essere attraversata in entrambe le direzioni. Si è
quindi rivolto all'impresa costruttrice Amadius che si fa pagare in
base al seguente opinabile criterio: il costo di una strada
corrisponde alla differenza di altitudine (in valore assoluto) fra le
due città facenti capo alla strada.

A compimento dell'opera due città saranno collegate se esiste una
serie di strade che parte dalla prima e arriva alla
seconda. L'ingegnere di corte è stato incaricato dal re di decidere
tra quali coppie di città costruire una strada in modo che tutte le
città siano collegate. Il costo di un progetto è dato dalla somma dei
costi delle strade da esso previste.

Calcola il minimo costo di un progetto. Calcola inoltre il massimo
costo di un progetto in cui ogni strada pianificata non possa essere
eliminata (pena lo scollegamento di due città).


% Assunzioni
\section*{Assunzioni}
\begin{itemize}[nolistsep, noitemsep]
\item Le altitudini delle città sono tutte distinte.
\item Il numero di città è $N\le 10^7$ e queste sono numerate da $0$ a
  $N-1$.
\item L'altitudine di ogni città è $\le 10^9$.
\item Le soluzioni sono entrambe numeri interi che rientrano nel range
  dei ``\texttt{long long int}'' (che consigliamo di usare).
\end{itemize}

% Subtasks
\section*{Subtask}
\begin{itemize}
\item \textbf{Subtask 0 [\phantom{1}5 punti]:} caso di esempio.
\item \textbf{Subtask 1 [17 punti]:} $N\le 1000$ e le altitudini sono $\le 10\,000$.
\item \textbf{Subtask 2 [26 punti]:} $N\le 10\,000$.
\item \textbf{Subtask 3 [29 punti]:} $N\le 10^6$.
\item \textbf{Subtask 4 [23 punti]:} nessuna limitazione specifica.
\end{itemize}


% Implementazione

\section*{Dettagli di implementazione}
Dovrai sottoporre esattamente un file con estensione \texttt{.c} o
\texttt{.cpp}. Questo file deve implementare la funzione
\texttt{stradizza} utilizzando il seguente prototipo.

\begin{verbatim}
void stradizza(int N, long long int* alt, long long int* Min, long long int* Max);
\end{verbatim}

Il parametro \verb|alt| è l'array di dimensione $N$ che contiene le
altitudini delle città. In \verb|alt[i]| è presente l'altitudine
dell'$i$-esima città.  I parametri \verb|Min| e \verb|Max| sono
puntatori alle variabili in cui la funzione deve scrivere
rispettivamente il minimo e il massimo possibile costo dei lavori, che
rispettano le richieste del re.

\subsection*{Funzionamento del grader di esempio}
Nella directory relativa a questo problema è presente una versione
semplificata (e un po' più lenta) del grader usato durante la
correzione che potete usare per testare le vostre soluzioni in
locale. Il grader di esempio legge i dati di input dal file
\file{input.txt}, a quel punto chiama la funzione \verb|bevi| che
dovete implementare, e scrive il risultato restituito dalla vostra
funzione sul file \file{output.txt}.

Nel caso vogliate generare un input, il file \file{input.txt} deve
avere questo formato:
\begin{itemize}
\item Prima riga: il numero di città $N$.
\item Seconda riga: $N$ interi distinti separati da spazi che indicano
  le altitudini delle città.
\end{itemize}

% Esempi
\section*{Esempio di input/output}

\esempio{
5

3 6 8 4 7

}{5 16}

\end{document}
